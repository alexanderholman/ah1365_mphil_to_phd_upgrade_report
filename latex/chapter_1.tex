\chapter{Introduction} \label{chapter:introduction}

Material interfaces are critical to the function and design of advanced solid-state systems. In semiconductors, they govern charge transport, carrier confinement, defect formation, and phase stability. In energy devices, they regulate ion exchange, chemical compatibility, and thermal response. In emerging heterostructures, interfaces can even induce novel behaviours not present in the constituent materials; such as encapsulated metastable phases or engineered Dirac cones. As device dimensions shrink and surface-to-volume ratios rise, the influence of interfaces grows, placing renewed emphasis on their accurate prediction and control. 

Yet despite their importance, predicting which interfaces are energetically favourable remains a significant challenge. Interface stability is not solely determined by bulk lattice matching, but also by local bonding, chemical inhomogeneity, and atomic-scale reconstruction. The structural space is vast: each pair of surfaces can be stacked with arbitrary lateral shifts, terminations, and alignments, resulting in a combinatorial explosion of configurations. Strain relaxation, interlayer interactions, and charge transfer further complicate the energy landscape. Accurately assessing favourability across this space demands a balance between resolution and scale that conventional methods struggle to achieve. 

First-principles techniques such as Density Functional Theory (DFT) remain the standard for computing accurate interfacial energies and relaxations. However, their high computational cost, scaling cubically with system size, renders exhaustive sampling intractable for realistic supercells, which often comprise hundreds of atoms. This bottleneck precludes broad screening efforts, especially when exploring multiple chemistries, orientations, and stacking arrangements. As such, DFT is best suited to final validation rather than initial discovery. 

To address this, the present project explores the use of machine-learned interatomic potentials (MLPs) for scalable interface screening. MLPs are data-driven models trained to reproduce DFT-level energies and forces, but at orders of magnitude lower cost. In particular, this work focuses on the MACE framework; an equivariant message-passing architecture that preserves rotational symmetries and captures local atomic interactions with high fidelity. By using MACE to rapidly evaluate large numbers of candidate interfaces, it becomes possible to rank structures by predicted stability and prioritise high-value configurations for subsequent DFT refinement. 

This approach supports a hierarchical modelling workflow in which DFT and MLPs are used in tandem: MLPs for breadth, and DFT for depth. It enables rigorous testing of whether MLPs can replicate DFT-derived favourability rankings across a diverse set of semiconductor systems, including homostructures (e.g. Si|Si) and heterostructures(e.g. Si|Ge, Ge|C). Moreover, it lays the foundation for further improvements through techniques such as delta-learning, structure-based prediction models, and surface-to-interface generalisation. 

The remainder of this report is structured as follows: 

\begin{itemize}
    \item \textbf{Chapter~\ref{chapter:background}} introduces the theoretical foundations and relevant literature, covering interface energetics, DFT formalism, and the development of MLPs.
    \item \textbf{Chapter~\ref{chapter:computational_investigation}} presents a benchmarking study comparing MLP and DFT predictions of interface favourability, including statistical correlation analysis and system-specific case studies.
    \item \textbf{Chapter~\ref{chapter:next_steps}} outlines planned extensions to the current methodology, including the development of delta-corrected potentials and predictive models linking surface properties to interface behaviour.
    \item \textbf{Chapter~\ref{chapter:conclusions}} concludes with a discussion of findings, limitations, and broader implications for autonomous materials modelling and the design of functional heterostructures.
\end{itemize}

Through this integrated modelling framework, the project seeks to reduce the cost and uncertainty of interface prediction while preserving the fidelity needed for scientific insight and technological relevance. 

\section{Motivation for the Research} \label{section:motivation} 

Interfaces rarely form under ideal conditions. Instead, they emerge through complex, often non-equilibrium processes such as high-temperature annealing, epitaxial growth, and chemically driven surface reconstructions. These dynamics frequently trap systems in metastable configurations or generate local disorder, resulting in atomic arrangements that deviate significantly from those predicted by ideal lattice-matching rules. Such complexity challenges efforts to model interface behaviour using conventional approaches alone. 

While Density Functional Theory (DFT) provides a reliable route to total energy prediction and structural relaxation, its unfavourable scaling with system size ($\mathcal{O}(N^3)$ in electron count) severely limits its use in high-throughput workflows. In particular, DFT becomes prohibitive when applied to interface supercells exceeding a few hundred atoms, as required to capture long-range strain, multi-layer reconstruction, or compositional variation. Moreover, standard DFT approximations, such as semi-local exchange-correlation functionals, may fail to capture interfacial band alignments, polarisation effects, or weak van der Waals interactions critical to layered systems. 

Machine-learned interatomic potentials (MLPs) offer a promising route to overcoming these limitations. By learning to emulate DFT-level energies and forces, MLPs like MACE provide a means to conduct large-scale structure relaxations and energy evaluations at dramatically reduced cost. This enables exploration of broader configurational spaces, including low-symmetry systems and extended defects, that would otherwise remain intractable. However, this advantage is only realised if the models generalise well across diverse interfacial environments. 

The core motivation of this work, therefore, is to assess whether MLPs can replicate the DFT-derived ordering of interface favourability, defined here in terms of relaxed formation energy, across a wide range of semiconductor heterostructures. The benchmarking in Chapter~\ref{chapter:computational_investigation} demonstrates that MLPs can often preserve the relative rankings of DFT energies, especially in chemically similar systems. However, notable discrepancies arise in cases such as pure-Si interfaces, where model performance degrades, prompting the need for refinement or retraining. These results motivate the delta-learning strategies explored in Chapter~\ref{chapter:next_steps}, which aim to correct systematic MLP-DFT residuals through supervised learning on interface-specific error distributions. 

In addition to correcting prediction accuracy, this research investigates whether interface favourability can be inferred directly from structural or surface-level features. Chapter~\ref{chapter:next_steps} proposes the use of learned models that map surface properties to interfacial energies, potentially bypassing the need for full supercell construction. This complements the structure-generation pipelines, based on ARTEMIS, that systematically enumerate viable interface candidates across various stacking shifts and Miller plane pairings. 

Ultimately, this project is motivated by the need for more efficient and predictive tools to accelerate the design of functional heterostructures. By embedding MLPs within automated screening workflows and augmenting them with delta-learning corrections and structure-based heuristics, it becomes possible to bridge the gap between physical realism and computational tractability. In doing so, this research contributes to the broader aim of enabling scalable, data-driven exploration of complex interfacial systems; particularly in regimes where conventional symmetry-based rules or bulk-derived heuristics break down. 

\section{Research Objectives} \label{section:research_objectives} 

This project aims to evaluate and enhance the use of machine-learned interatomic potentials (MLPs) for predicting the relative stability of material interfaces. By replacing expensive density functional theory (DFT) calculations in the early stages of interface screening, MLPs offer a route to scalable evaluation across diverse chemical systems and stacking configurations. 

Interfaces govern the functional behaviour of materials in microelectronics, energy storage, and optoelectronics, mediating phenomena such as band alignment, carrier confinement, and local defect formation. However, the configurational landscape of interface structures is combinatorially large. Even for lattice-matched systems, variations in Miller planes, interfacial shifts, and terminations can lead to significant changes in formation energy and electronic response. Traditional DFT methods, while accurate, are computationally prohibitive when applied to hundreds or thousands of such configurations. 

This work builds on the integration of MLPs, particularly the MACE model, into high-throughput interface generation and evaluation pipelines. It focuses not only on energy prediction, but also on favourability ranking, robustness to system-specific errors, and hybrid workflows where MLPs are used as a pre-screening stage ahead of final DFT validation. 

\paragraph{The research addresses the following core objectives:} 

\begin{enumerate}
    \item \textbf{Benchmark MLP-predicted rankings against DFT across diverse interfaces.} Using MACE-evaluated relaxations and energies, this study quantifies how well MLPs preserve the DFT ordering of interface favourability across multiple group-IV homostructures and heterostructures.
    \item \textbf{Diagnose system-specific discrepancies and explore delta-learning corrections.} Particular attention is given to cases, such as pure-Si systems, where MLP rank fidelity degrades. Chapter~\ref{chapter:next_steps} proposes training delta models to correct these systematic energy errors.
    \item \textbf{Develop structure-based predictors of interface energy.} Surface and slab-level features (e.g. orientation, element type, symmetry, strain) will be used to train models capable of predicting favourability prior to full interface assembly, reducing reliance on supercell enumeration.
    \item \textbf{Integrate MLPs and predictors into automated workflows.} Tools such as ARTEMIS and RAFFLE will be extended to incorporate MLP-based scoring or filtering, enabling efficient identification of favourable configurations before expensive DFT refinement.
\item \end{enumerate}

In the long term, this approach supports scalable interface screening, facilitates the design of functional heterostructures, and lays groundwork for autonomous modelling pipelines. It targets regimes where standard coherence rules or bulk-derived heuristics fail, providing a data-driven alternative grounded in atomic-level energetics. 

\section{Scientific and Practical Significance} \label{section:scientific_and_practical_significance} 

Material interfaces are central to the behaviour of functional solids, especially at the nanoscale where surface and interfacial effects dominate bulk properties. In semiconductors, they define charge transport pathways by setting potential barriers and band offsets. In batteries, they regulate ionic conductivity, chemical stability, and degradation kinetics. In photovoltaics and quantum devices, interfacial dipoles, tunnelling channels, and confinement effects underpin key device operations. 

Subtle differences in stacking, registry, or atomic reconstruction can shift critical electronic properties by several electronvolts; altering conductivity, carrier lifetimes, or activation barriers. These shifts emerge from a complex interplay of interfacial strain, chemical bonding, and electronic redistribution, and are sensitive to both global structure and local environment. As such, predicting interface stability and functionality from atomic configuration alone remains a demanding task. 

While Density Functional Theory (DFT) offers a robust ab initio framework for computing interfacial energies and electronic properties, its computational cost scales poorly with system size. Realistic supercells, often comprising hundreds of atoms, are therefore difficult to study in large numbers. DFT is well-suited to final validation, but ill-suited to exhaustive exploration across configurational space. 

Machine learning (ML) models, particularly machine-learned interatomic potentials (MLPs), provide a way forward. When trained on DFT data, MLPs can reproduce formation energies and atomic forces at orders-of-magnitude lower cost, making them suitable for broad screening applications. Their integration into automated workflows, using tools such as ARTEMIS and RAFFLE, enables systematic generation and evaluation of interface candidates across crystallographic orientations, stacking shifts, and chemical combinations. 

Importantly, the utility of ML does not stop at static energetics. Many applications require insight into how interfaces behave under external stimuli; electric fields, temperature gradients, or chemical exposure. Emerging interfacial phases, such as field-tunable dielectric layers or metastable heterostructures, challenge traditional heuristics and call for more adaptive and scalable modelling approaches. 

This project contributes to that broader effort. It assesses the ability of MLPs, specifically MACE, to recover DFT-derived favourability rankings, identifies where discrepancies arise, and proposes delta-learning corrections to address them. It also lays groundwork for predictive models that estimate interface properties from top and bottom slabs, enabling pre-screening before full interface construction. By focusing on group-IV semiconductor systems and both homostructure and heterostructure cases, this work targets a class of technologically relevant interfaces where modelling uncertainty remains high and conventional rules often fail. 

Ultimately, this research supports the advancement of data-driven, scalable methods for interface discovery; bridging the gap between computational feasibility and the physical fidelity required for next-generation materials design. 
\chapter{Background and Literature Review}
\label{chapter:background}

\section{Defining Material Interfaces}
\label{section:material_interfaces}

This subsection establishes a foundational understanding of material interfaces, the boundary regions where two
distinct phases, often different materials, meet. These interfaces are critical in determining the structural,
electronic, and thermal behaviour of composite systems. In solid-state devices, subtle atomic arrangements and local
electronic environments at interfaces can profoundly influence macroscopic properties. This is especially true in
modern micro- and nanoscale technologies, where the continual miniaturisation of components has shifted the paradigm
from bulk-dominated behaviour to interface-dominated performance. As devices shrink, interfaces no longer represent
isolated regions but instead constitute a significant fraction of the material, making their precise definition and
understanding essential for predictive modelling and materials design.

\subsection{What is a material interface in the context of this research?}

A material interface is the region where two distinct solids meet and form a junction, commonly arising in
heterostructures, grain boundaries, or thin film systems. These regions are not only geometrical boundaries but
exhibit unique atomic, electronic, and thermal properties that distinguish them from the bulk phases of the parent
materials. The local structure at the interface can include strain fields, atomic reconstruction, and intermixing of
constituent elements, all of which can significantly influence the behaviour of the composite system.

In this research, material interfaces are studied predominantly through \textit{ab initio} methods, particularly
Density Functional Theory (DFT), with a focus on predicting and analysing the structural and energetic properties of
2D|2D and 2D|3D junctions. Interfaces are modelled as systems with finite thickness that bridge two crystalline
lattices. Depending on their geometric orientation, interfaces are categorised as either \emph{stacked}, where a 2D
layer rests atop a substrate, or \emph{lateral}, where edge-on growth leads to in-plane bonding between dissimilar
materials.

A critical motivation for interface studies lies in their role in determining key device properties such as band
alignment, carrier transport, and thermal resistance. Band offsets at interfaces, often deviating from predictions
by simple models like Anderson\rqs rule, are governed by local charge redistribution, dipole formation, and
interface-specific states. The emergence of new electronic phases at interfaces, such as Ohmic contact formation or
type-II band alignments, necessitates accurate modelling beyond bulk considerations.

Several methodologies have been developed to address the challenges of constructing realistic interface models. The
ARTEMIS tool enables high-throughput generation of interface structures through lattice matching and surface
termination selection, allowing systematic exploration of the complex interfacial phase space. Similarly, the RAFFLE
method uses statistical learning and void-filling algorithms to construct interface phases that are energetically
favourable and representative of plausible morphologies.

Interfaces also significantly influence phonon transport. They can reduce thermal conductivity compared to bulk
values; an effect exploitable in thermoelectric applications. Interfacial scattering, acoustic impedance mismatch,
and atomic-scale disorder all contribute to this behaviour, reinforcing the interface\rqs role as a functional
component rather than a passive boundary.

Thus, in the context of this research, a material interface is not merely a geometric boundary but an emergent,
functionally distinct region. It governs critical physical properties, demands precise structural modelling, and
serves as a fertile ground for material innovation and device engineering.

\subsubsection{Types of interfaces}

Material interfaces can be broadly categorised by their crystallographic orientation, dimensionality, and degree of
lattice registry. These classifications are instrumental in understanding interfacial phenomena and in designing
simulation protocols that are physically realistic and computationally tractable.

\paragraph{By geometry: stacked vs lateral.}
Stacked interfaces occur when one material is deposited atop another, forming an out-of-plane junction. This is
characteristic of many van der Waals heterostructures, where weak interlayer forces enable a range of stacking
configurations and rotational alignments. In contrast, lateral interfaces arise when two materials are grown
side-by-side, often leading to in-plane covalent bonding at the junction. These interfaces are especially relevant
for in-plane heterostructures such as MoS$_2$|WS$_2$ or graphene-based edge-junctions.

\paragraph{By coherence: coherent, semi-coherent, incoherent.}
Interfaces can also be distinguished by the degree of lattice match across the boundary:

\begin{itemize}
    \item \textbf{Coherent interfaces} exhibit perfect registry between atomic planes, typically occurring when lattice
    mismatch is negligible. These interfaces preserve the translational symmetry across the junction and are often
    found in homoepitaxial or pseudomorphic systems.
    \item \textbf{Semi-coherent interfaces} accommodate lattice mismatch through periodic misfit dislocations. These
    dislocations relieve interfacial strain while maintaining partial registry, allowing larger domains of coherency
    interspersed with localised defects.
    \item \textbf{Incoherent interfaces} display no periodicity match across the interface, leading to structurally
    disordered junctions. These are often found in polycrystalline or amorphous/crystalline composites and can act as
    strong phonon or electron scattering centres.
\end{itemize}

\paragraph{By chemical contrast: heterojunctions vs homojunctions.}
Chemical composition also plays a defining role. \emph{Homojunctions} involve structurally similar regions with
subtle variations (e.g. doping or strain-modulated domains), while \emph{heterojunctions} involve distinct materials
with differing electronegativity, band gaps, and ionic/covalent character. Heterojunctions are central to
semiconductor devices, solar cells, and catalysis, often giving rise to emergent interfacial dipoles, states, or
reconstructed bonding motifs.

These classifications are not mutually exclusive; real interfaces may be simultaneously lateral and semi-coherent,
or stacked but chemically graded. Understanding these distinctions is essential when generating or interpreting
simulation results for both idealised and realistic interfacial systems.

\subsection{What physical or electronic properties are commonly affected by interfaces?}

Material interfaces influence a wide array of physical and electronic properties, often determining the functional
behaviour of composite systems and devices. At the atomic level, interfaces introduce discontinuities in crystal
symmetry, strain fields, and chemical potential that can lead to emergent phenomena not observed in bulk
constituents. These effects are particularly pronounced in systems involving 2D materials, complex oxides, or
strongly correlated electron systems.

\subsubsection{Band Alignment and Electronic Transport}

Perhaps the most critical electronic property influenced by interfaces is the band alignment between materials.
Misalignment of conduction and valence band edges creates energy barriers or wells that determine whether an
interface behaves as a Schottky barrier, Ohmic contact, or heterojunction. These alignments are crucial for device
operation in field-effect transistors, diodes, and photovoltaics. Interfacial dipoles, charge transfer, and
electronic reconstruction can shift the effective band edges, often invalidating simple rules like Anderson\rqs model.

\subsubsection{Charge Carrier Dynamics}

Interfaces often host localised states arising from atomic mismatch, bonding reconstruction, or impurity
segregation. These can act as traps or recombination centres, significantly affecting carrier lifetimes and
mobility. In 2D|3D systems, tunnelling, leakage, or enhanced recombination may occur depending on the barrier height
and band offset at the junction.

\subsubsection{Dielectric Response and Permittivity}

The dielectric properties of materials can be strongly altered at interfaces. In systems like \ce{CaCu3Ti4O12},
colossal permittivity arises not from the bulk crystal but from insulating grain boundaries and a semiconducting
interior, exemplifying the Maxwell-Wagner effect. Interface-induced polarisation, charge accumulation, and ionic
displacements all contribute to enhanced dielectric responses.

\subsubsection{Thermal Conductivity}

Thermal transport across interfaces is typically suppressed due to phonon scattering, mass disorder, and acoustic
impedance mismatch. This results in thermal boundary resistance (or Kapitza resistance), a key factor in
thermoelectric design and heat management in microelectronics. In 2D heterostructures, the interface often acts as a
dominant phonon scattering site, reducing the effective thermal conductivity below that of either constituent layer.

\subsubsection{Structural Stability and Phase Formation}

Interfaces can stabilise new or metastable phases not accessible in bulk. For instance, the interface between
\ce{BaTiO3} and \ce{SiO2} has been shown to promote the spontaneous formation of the fresnoite phase \ce{Ba2TiSi2O8},
altering both electronic and mechanical response. A more nuanced example is the graphene|\ce{MgO} interface: while
monolayer MgO confined between graphene layers often adopts a rocksalt-like structure, this is not simply a case of
one phase being energetically favoured. Instead, the local environment, including charge transfer and encapsulation,
stabilises the rocksalt phase relative to the expected hexagonal phase in unsupported monolayers. However, experimental
and theoretical studies suggest that a mixed or hybrid phase may also eme.g. with phase stability dependent on
thickness and interface strain.

\subsubsection{Mechanical Properties and Elastic Moduli}

Mechanical properties such as bulk modulus, shear strength, and adhesion energy are modified at interfaces due to
altered bonding environments and lattice mismatch. These variations are important in the design of composite
materials, flexible electronics, and nanoindentation studies.

\subsubsection{Optical Properties and Absorption}

Interfaces affect the optical absorption spectrum by introducing new electronic states and modifying the joint
density of states. This can lead to bandgap narrowing or broadening, shifts in refractive index, and enhancement of
excitonic effects; features exploited in photodetectors and photovoltaic devices.

\paragraph{In Summary} material interfaces act as multifunctional domains that modulate a broad spectrum of material
properties. Their control and prediction are vital for the design of next-generation electronic, optoelectronic, and
thermal devices.

\subsection{Why are interfaces critical in material engineering, electronics, or energy systems?}

Interfaces are not merely passive boundaries between materials; they are active, functional regions that frequently
control and even dominate the physical properties of the system. As materials and devices are engineered at ever
smaller scales, the role of interfaces becomes more pronounced. In conventional bulk systems, interfaces may be
limited to grain boundaries or inter-phase regions. However, in nanomaterials, layered heterostructures, and
device-scale systems, interfacial regions can constitute a non-negligible fraction of the total volume or surface
area. Their influence is therefore not ancillary but foundational to system performance.

In materials engineering, interfaces govern a wide array of properties including adhesion, corrosion resistance,
mechanical strength, and diffusion kinetics. The strength and ductility of composite systems or coatings often
depend on interfacial bonding and lattice compatibility. Tailoring interfacial chemistry or morphology, through
methods such as surface functionalisation, alloying, or atomic layer deposition, is often essential for achieving
desired macroscopic performance metrics.

In electronic devices, interfaces critically influence charge transport, band alignment, and contact resistance.
Semiconductor heterojunctions, metal-semiconductor contacts, and dielectric boundaries are fundamental to the
operation of transistors, diodes, and capacitors. Localised changes in atomic registry, roughness, or defect density
at the interface can significantly alter tunnelling behaviour, barrier heights, and mobility. This is especially
true in 2D|3D systems, where interfacial dipoles, charge redistribution, and electrostatic reconstruction result in
complex electronic responses not captured by bulk models. As devices shrink to the nanometre scale, quantum
confinement and interfacial electrostatics become dominant factors, necessitating atomically precise interface control.

Energy systems are similarly interface-dominated. In lithium-ion batteries, for example, the solid-electrolyte
interphase (SEI) acts as a metastable interfacial layer that dictates both stability and charge transport. The
performance and longevity of solid-state batteries, fuel cells, and thermoelectric devices depend on the engineered
properties of interfaces, which affect ionic conductivity, catalytic activity, thermal transport, and electronic
insulation. In photovoltaics, interfacial band offsets between donor and acceptor layers, or between absorber
materials and charge transport layers, govern charge separation efficiency and open-circuit voltage.

Overall, interfaces are not incidental features but active design parameters. They serve as sites of emergent
phenomena, such as interfacial polarisation, bandgap renormalisation, and phonon scattering, that cannot be
predicted from bulk properties alone. As such, mastering interfacial science is a gateway to developing
high-performance materials and devices.

\subsection{How do crystallographic orientations, phase boundaries, or defects influence interfaces?}

Material interfaces are regions where two or more distinct phases meet, and their atomic-level structure plays a
critical role in determining the physical, electronic, and thermal properties of the overall system. These
interfaces are not merely passive boundaries; rather, they are active regions that can be engineered for specific
functionality, particularly in devices involving electronic, optoelectronic, and energy materials. The structural
features that influence interface behaviour include crystallographic orientation, the nature of the phase boundary,
and the presence of defects.

\subsubsection{Crystallographic Orientation}

The relative orientation of crystals on either side of an interface can significantly affect the atomic registry,
bond continuity, and symmetry breaking across the interface. Epitaxial alignment, where lattice planes across the
interface maintain a well-defined relationship, typically results in low-strain, coherent interfaces. In contrast, a
mismatch in orientation can lead to misfit dislocations or interfacial strain, influencing charge carrier transport,
interfacial states, and mechanical stability. Orientation also affects the formation energy and electronic
reconstruction at the interface, which can lead to phenomena like interface-induced states or altered band alignments.

\subsubsection{Phase Boundaries}

When the interface forms between distinct structural phases, such as cubic and hexagonal polymorphs, unique bonding
geometries and strain fields arise. For instance, in layered 2D|3D systems like \ce{HfS2}|\ce{HfO2}, the band
alignment changes depending on whether the materials are stacked or laterally connected, with lateral interfaces
showing increased reconstruction due to bonding mismatch between the different crystalline motifs. These
reconstructions are not merely geometric; they result in different electrostatic and electronic behaviours,
influencing band offsets and charge transfer dynamics.

\subsubsection{Defects at Interfaces}

Defects, including vacancies, interstitials, dislocations, and grain boundaries, are often intrinsic to interface
formation and can profoundly modify interfacial properties. They may serve as scattering centres, recombination
sites, or sources of local doping. In grain boundaries of polycrystalline materials, for example, variations in
local stoichiometry and strain can create regions of localised metallicity or enhanced dielectric response, as shown
in systems exhibiting colossal permittivity. In some systems, such as CCTO (\ce{CaCu3Ti4O12}), interfacial defects
and microstructural features have been shown to dominate the macroscopic dielectric properties.

In computational studies, predicting the impact of such defects requires careful treatment of their spatial
distribution, charge state, and interaction with surrounding lattice atoms. Modern approaches like the RAFFLE
algorithm allow for machine learning-driven structure prediction that accounts for void-filling and local bonding
environments, enabling more realistic models of defect-laden interfaces.

\subsection{Examples of important technological applications involving interfaces}

Material interfaces, where two distinct phases or materials meet, play a pivotal role in determining the performance
and function of a wide array of modern technologies. These regions often exhibit structural, electronic, or thermal
properties that differ markedly from those of the bulk constituents. Depending on their nature, interfaces may
enhance or impede material performance. This subsection surveys several application domains in which understanding
and engineering interfacial properties is essential.

In semiconductor devices, the interface between the gate dielectric and the semiconducting channel is a critical
design parameter. As traditional \ce{SiO2} dielectrics are replaced by high-$\kappa$ materials such as \ce{HfO2} to
meet miniaturisation demands, understanding charge transfer and band alignment across interfaces like \ce{HfS2}|
\ce{HfO2} becomes essential for ensuring device reliability and performance. Notably, in such 2D|3D
heterostructures, the type and strength of bonding (e.g. van der Waals vs. chemical) directly influences charge
redistribution and can cause significant deviation from band alignments predicted by Anderson\rqs rule.

In layered 2D electronics, especially those using transition metal dichalcogenides (TMDCs), the stacking geometry,
layer orientation, and interfacial disorder all strongly influence the band alignment. These factors affect carrier
dynamics, optical transitions, and switching behaviour. Traditional models such as Anderson\rqss rule have proven
insufficient; corrections using terms such as $\Delta E_\Gamma$ and $\Delta E_{\text{IF}}$ have been developed to
capture hybridisation and interface dipole contributions more accurately. These corrections allow for predictive
modelling of heterostructure behaviour and support targeted band offset engineering.

Interfaces also govern performance in energy storage systems. In all-solid-state batteries, the interface between
electrode and solid electrolyte dictates ion mobility, chemical stability, and space charge layer formation. Such
interfacial processes can lead to resistive interphases or dendrite formation, reducing battery life and safety.
Density Functional Theory (DFT) modelling has shown that interfacial reconstructions and electrostatic gradients can
dominate device behaviour.

In thermoelectric materials, interfaces are employed to decouple electrical and thermal transport. Interface-induced
phonon scattering suppresses thermal conductivity while preserving electronic conductivity; enhancing the
thermoelectric figure of merit, $ZT$. Layered TMDCs and heterostructures with nanostructured interfaces exploit this
mechanism to improve thermal management.

Finally, in optoelectronics and photovoltaics, interfaces play a central role in determining band bending, exciton
dissociation, and charge carrier collection. In oxide-based devices, such as proposed all-oxide solar cells (e.g.
\ce{CaO}|(\ce{Sn}:\ce{Ca})$_{7:1}$\ce{O}|\ce{TiO2}), interface energetics must be carefully tuned to support efficient
charge separation and transport. The formation of intermediate phases or interfacial dipoles can critically affect
performance, necessitating first-principles modelling for reliable device design.

Across these examples, it is evident that interfaces are not passive boundaries, but active, tunable regions that
critically determine macroscopic functionality. Their predictive modelling, through DFT, machine learning
potentials, or interface-specific structure search tools like ARTEMIS, remains a central challenge in modern
materials design.


\section{Conventional Prediction Methods}
\label{section:conventional_methods}

Understanding and predicting the behaviour of material interfaces requires a firm foundation in computational
methods that are rooted in quantum and statistical mechanics. This section introduces the core approaches
historically used to model interfaces at the atomic and mesoscale level, focusing primarily on first-principles
methods and classical simulation techniques. Although each of these methods offers powerful insights, their
accuracy, computational cost, and suitability for interface modelling vary substantially.

A primary tool in this field is \textit{Density Functional Theory} (DFT), a quantum-mechanical framework capable of
describing electronic structure with high precision. DFT is well-suited to calculating key interfacial properties
such as formation ene.g. charge redistribution, and band alignment, making it an essential component of most modern
interface studies. Within this domain, software packages such as \textsc{VASP} have become widely adopted for their
efficient implementations of the Kohn-Sham formalism.

However, DFT\rqss cubic scaling with system size makes it computationally prohibitive for large or disordered interface
models. As a result, researchers also turn to \textit{classical methods}, such as \textit{Molecular Dynamics} (MD)
and \textit{Kinetic Monte Carlo} (KMC). MD enables the simulation of dynamic processes and thermal fluctuations over
nanosecond timescales, whereas KMC offers access to much longer time evolution by modelling stochastic transitions
between states. Each method has its own niche---MD is adept at capturing thermal effects and atomic rearrangements,
while KMC is particularly valuable for studying diffusion and growth kinetics over experimental timescales.

These conventional approaches provide the groundwork for predicting critical interfacial properties---including
adhesion ene.g. charge transfer behaviour, and electron or phonon transport. They remain indispensable despite their
limitations, particularly when used in tandem. This section surveys these methods in turn, addressing their
theoretical underpinnings, practical implementations, and their relative advantages and constraints in the context
of interface science.

\subsection{What are traditional computational approaches to modelling interfaces?}

Traditional computational approaches to modelling interfaces have historically focused on atomistic methods grounded in
quantum mechanics or statistical physics. The most widely adopted of these are \textit{first-principles} or
\textit{ab initio} methods, particularly Density Functional Theory (DFT), as well as Molecular Dynamics (MD) and
Kinetic Monte Carlo (KMC). Each offers a different perspective on interfacial structure, energetics, and evolution.

\subsubsection{Lattice Matching and Geometry-Driven Construction}

A foundational step in many interface modelling workflows is lattice matching between two crystalline materials.
This involves aligning compatible Miller planes to minimise strain and identify viable supercell geometries. The
method introduced by Zur et al.~(1984) laid the groundwork for many subsequent algorithms used in tools such as
\textsc{METADISE} and \textsc{ARTEMIS}, which systematically enumerate low-strain combinations, slips, terminations,
and atomic registries. These approaches assume idealised surfaces and often neglect more complex processes such as
surface reconstructions, chemical interdiffusion, or local disorder.

\subsubsection{Density Functional Theory (DFT)}

DFT remains the most accurate and widely used quantum mechanical method for predicting interfacial properties. It
solves the Kohn--Sham equations to determine the electronic ground state, providing access to total energies, charge
densities, and band edge positions. DFT can reliably predict key quantities such as adhesion ene.g. charge transfer,
and local band alignment. However, it scales cubically with system size ($\mathcal{O}(N^3)$), limiting its
application to relatively small supercells and coherent interfaces. For disordered systems or long-range
relaxations, DFT becomes prohibitively expensive.

\subsubsection{Molecular Dynamics (MD)}

MD simulates the atomic trajectories under classical Newtonian mechanics, governed by interatomic potentials.
Classical MD is well-suited to studying thermal processes such as diffusion, defect migration, and strain relaxation
on picosecond to nanosecond timescales. It can handle much larger system sizes than DFT but sacrifices electronic
accuracy, particularly in capturing bond rearrangements or charge redistribution. Reactive MD and ab initio MD (e.g.
Car-Parrinello) offer improved fidelity at increased computational cost.

\subsubsection{Kinetic Monte Carlo (KMC)}

KMC provides a probabilistic approach to simulate rare events and long-term evolution, such as grain growth,
interdiffusion, or defect clustering. It models a system as a sequence of stochastic transitions between predefined
states, with rates typically derived from DFT or MD simulations. KMC can simulate processes on experimentally
relevant timescales (up to seconds), but lacks atomic-resolution trajectories and requires a comprehensive catalogue
of rate-limiting events.

\subsubsection{Summary}

Together, these methods form the traditional computational toolkit for interface modelling. DFT provides
high-accuracy predictions of interfacial energies and electronic structure; MD captures time-resolved atomistic
behaviour; and KMC enables long-timescale simulations of kinetic evolution. While powerful individually, these
methods are increasingly applied in tandem to span multiple length and time scales.

\subsection{Density Functional Theory (DFT)}

Density Functional Theory (DFT) is a first-principles quantum mechanical method used extensively to model the
electronic structure of condensed matter systems, including molecules, solids, and, critically, material interfaces.
It provides a variational framework in which the ground-state properties of a many-electron system can be obtained
from a functional of the electron density $\rho(\mathbf{r})$, rather than from the many-body wavefunction
$\Psi(\mathbf{r}_1, \dots, \mathbf{r}_N)$. This simplification, formalised through the Hohenberg-Kohn theorems, underpins the
modern implementation of DFT via the Kohn-Sham approach, where the interacting system is replaced by an auxiliary
system of non-interacting electrons subject to an effective potential that captures Coulomb, exchange, and correlation
interactions.

The total Kohn-Sham energy is given by:
\begin{equation}
    E_{\mathrm{KS}}[\rho] = T_s[\rho] + \int V_{\mathrm{ext}}(\mathbf{r}) \rho(\mathbf{r}) \, \mathrm{d}\mathbf{r} + \frac{1}{2} \int \int \frac{\rho(\mathbf{r}) \rho(\mathbf{r'})}{|\mathbf{r} - \mathbf{r'}|} \, \mathrm{d}\mathbf{r} \, \mathrm{d}\mathbf{r'} + E_{\mathrm{xc}}[\rho],
\end{equation}
where $T_s[\rho]$ is the kinetic energy of the non-interacting electrons, $V_{\mathrm{ext}}$ is the external (ionic)
potential, and $E_{\mathrm{xc}}[\rho]$ is the exchange-correlation functional, the exact form of which remains unknown.

Common approximations include the Local Density Approximation (LDA), Generalised Gradient Approximation (GGA), and
hybrid functionals. While LDA assumes a locally uniform electron gas and often leads to overbinding, GGA
incorporates density gradients and improves accuracy for molecular and inhomogeneous systems. Hybrid functionals,
such as HSE06, incorporate a portion of exact exchange and offer improved bandgap and localisation predictions at
significantly greater computational cost.

DFT has become foundational in materials modelling due to its ability to predict not only total energies and
optimised geometries, but also derived properties including electronic band structures, charge densities, dielectric
responses, phonon spectra, and interfacial energetics. Its applications to interface modelling, however, are
computationally constrained. Systems containing hundreds of atoms, often required to model misfit dislocations,
strain relief, and interfacial disorder, can become prohibitively expensive, with computational scaling on the order
of $\mathcal{O}(N^3)$ with system size.

\subsubsection{VASP and Practical Implementations}

All DFT simulations in this project are conducted using the Vienna Ab initio Simulation Package (VASP), a widely
used plane-wave-based implementation of Kohn-Sham DFT. VASP employs the projector-augmented wave (PAW) method to
model core-valence interactions, and supports a variety of exchange-correlation functionals including GGA-PBE and
hybrid HSE06. Structural relaxations are typically performed with conjugate gradient algorithms until residual
forces fall below a chosen threshold (e.g. 0.01 eV/\AA).

Periodic boundary conditions are employed in all three spatial directions, necessitating the introduction of vacuum
regions for slab models and sufficient supercell dimensions to minimise image-image interactions in interface
simulations. For phonon and dielectric property calculations, VASP supports both finite displacement and Density
Functional Perturbation Theory (DFPT) approaches, though care must be taken when treating low-symmetry systems or
low-frequency modes.

The simulation input is controlled via the INCAR, POSCAR, KPOINTS, and POTCAR files, with convergence typically
assessed by k-point density and energy cutoff tests. Pulay and Broyden mixing schemes are used for charge density
convergence, and dipole corrections may be employed for asymmetric slab geometries.

While VASP provides accurate and robust results for interfacial modelling, it remains computationally intensive. For
extended or complex interfacial systems, this motivates the introduction of machine-learned potentials (MLPs), which
aim to reproduce DFT accuracy at substantially lower cost; enabling exploration of larger supercells, disordered
motifs, and thermodynamic ensembles that are otherwise intractable.

\subsection{Other Methods}

In addition to Density Functional Theory (DFT), two other classes of atomistic modelling, Molecular Dynamics (MD)
and Kinetic Monte Carlo (KMC), have proven valuable for exploring material interfaces, particularly when temporal
evolution or thermal effects are of interest.

\subsubsection{Molecular Dynamics}

Molecular Dynamics is a time-resolved simulation method in which the trajectories of atoms are computed by
numerically integrating Newton\rqss equations of motion. Forces on each atom are derived either from empirical
interatomic potentials (classical MD) or quantum-mechanical calculations (ab initio MD, e.g. Car-Parrinello or
Born-Oppenheimer approaches). In interface science, MD is used to study thermal stability, strain relaxation,
diffusion, and defect evolution at finite temperatures.

% todo add citation

One advantage of MD is its capacity to resolve dynamic processes at the atomic level, such as dislocation glide,
grain boundary motion, or interfacial melting. However, timescales are severely limited: typical simulations span
nanoseconds, far shorter than the seconds to minutes often relevant in experiments. Classical MD can scale to
millions of atoms but sacrifices accuracy; ab initio MD is more accurate but typically limited to fewer than 1,000
atoms and short timescales due to computational cost.

% todo add citation

For interface modelling, MD is particularly useful when studying thermally activated restructuring, interdiffusion,
and local vibrational behaviour, especially under conditions such as annealing or irradiation. It also serves as a
preliminary stage in workflows involving machine-learned potentials, providing dynamical data for training.

\subsubsection{Kinetic Monte Carlo (KMC)}

Kinetic Monte Carlo methods simulate the time evolution of materials by selecting thermally activated events (such
as diffusion, adsorption, or defect migration) based on their rates, derived from either experimental data or
calculated energy barriers. Rather than resolving individual atomic vibrations, KMC focuses on probabilistic
transitions between states, making it well-suited to model slow processes over long timescales, up to seconds or beyond.

KMC has found particular utility in modelling interface evolution under conditions of growth, ion exchange,
oxidation, and defect diffusion. In grain boundary studies, for example, KMC can capture the gradual restructuring
of interfaces under chemical potential gradients or during irradiation recovery.

Its main limitation is the requirement for a comprehensive event catalogue with well-characterised energy barriers
and transition rates. These are often obtained from DFT or empirical fitting, and inaccuracies in this input can
compromise the reliability of the predictions. Furthermore, KMC lacks atomic-resolution trajectories and cannot
directly model phonon-mediated transport or structural vibrations.

\subsubsection{Summary}

Both MD and KMC complement DFT in the modelling of interfaces. While DFT offers high-fidelity energetics and
electronic structure, MD enables atomistic dynamics over nanoseconds, and KMC extends the accessible timescale to
macroscopic phenomena. As such, hybrid approaches are increasingly common, with DFT used to parameterise MD or KMC
models, or machine learning techniques introduced to accelerate sampling across scales.

\subsection{What are the advantages and limitations of DFT, MD, and KMC?}

Each computational method traditionally employed in the modelling of interfaces, Density Functional Theory (DFT),
Molecular Dynamics (MD), and Kinetic Monte Carlo (KMC), offers a distinct balance between accuracy, computational
efficiency, and physical scale. While often applied in isolation, their complementary regimes suggest the need for
hybrid or hierarchical approaches in the study of real interfaces.

\textbf{Density Functional Theory (DFT)} is the most widely used \textit{ab initio} method for predicting
atomic-scale properties at interfaces. It provides access to ground-state energies, relaxed geometries, charge
densities, and band alignments with a strong quantum mechanical foundation. Crucially, DFT is capable of modelling
materials across the periodic table and is the backbone of modern electronic structure prediction. Its limitations,
however, are twofold. Firstly, the method scales poorly with system size, typically as $\mathcal{O}(N^3)$ with
respect to the number of electrons or basis functions, making simulations involving more than a few hundred atoms
prohibitively expensive. Secondly, conventional exchange-correlation functionals (e.g. GGA) systematically
underestimate band gaps and cannot capture excited-state dynamics without costly extensions such as GW or
time-dependent DFT.

\textbf{Molecular Dynamics (MD)} offers time-resolved access to thermally activated processes and interfacial dynamics.
Classical MD, using empirical or machine-learned interatomic potentials, enables simulation of thousands to millions of
atoms over nanosecond to microsecond timescales. It is well-suited to studying lattice relaxation, phonon scattering,
or diffusion across interfaces. However, MD inherits limitations from its force field or potential model: empirical
potentials may not transfer well across bonding environments, while high-fidelity machine-learned potentials require
extensive and chemically diverse training data. Furthermore, standard MD remains constrained to accessible timescales,
making it ill-suited for capturing rare events such as defect migration, reconstruction, or long-term phase evolution.

\textbf{Kinetic Monte Carlo (KMC)} bridges the gap between atomistic fidelity and long-timescale evolution by
explicitly simulating stochastic processes with predefined transition rates. It is particularly useful for modelling
diffusion, growth, or reaction networks at interfaces. The method can span experimental timescales (seconds to hours),
offering a unique perspective on kinetic stability and metastability. The key limitation lies in its dependency on an
accurate catalogue of possible events and associated rates. These must be either calculated from first principles or
parametrised experimentally, introducing significant uncertainty and limiting transferability to novel systems.

\paragraph{In summary} DFT is unrivalled in electronic accuracy but severely limited in scale. MD extends spatial and
temporal reach at the cost of potential fidelity, while KMC pushes further still into kinetic regimes but demands
well-characterised reaction landscapes. The combined use of these methods, e.g. DFT-informed MD or KMC parametrised by
\textit{ab initio} barriers, has emerged as a powerful paradigm for interface modelling across scales.

\subsection{How are properties like adhesion energy or electron transport predicted?}

The accurate prediction of interfacial properties, such as adhesion energy and electron transport, is central to
understanding and designing functional materials. These properties emerge from complex interplays of bonding, charge
distribution, and band alignment at the interface, and are typically predicted through first-principles simulations
based on density functional theory (DFT). Increasingly, these predictions are augmented or accelerated using
machine-learned interatomic potentials (MLPs).

\textbf{Adhesion energy} is a thermodynamic quantity describing the work required to separate two materials at their
interface. It is commonly computed from the total energies of the isolated constituents and the relaxed interface
configuration. Within DFT, the adhesion energy $E_\mathrm{adh}$ is given by:

\[
    E_\mathrm{adh} = \frac{1}{A} \left( E_\mathrm{slab1} + E_\mathrm{slab2} - E_\mathrm{interface} \right),
\]

where $A$ is the interface area, and the energies refer to fully relaxed geometries of the individual slabs and the
interface supercell. For 2D|2D lateral heterostructures, where the interface is line-like rather than planar, a line
energy formulation in eV/\AA is typically used to account for the dimensionality of bonding.

\textbf{Electron transport} across interfaces is more nuanced and often requires evaluation of both band alignment and
carrier mobility. Band alignment can be predicted via several approaches. The simplest, Anderson\rqss rule, aligns vacuum
levels to estimate offsets from electron affinities or ionisation potentials. However, this often fails to capture
interface-specific effects such as charge transfer, built-in fields, or hybridisation. More accurate predictions
therefore rely on self-consistent DFT calculations that extract the local electrostatic potential across the
heterostructure to determine band offsets directly.

For example, the offset between conduction or valence band edges across an interface can be obtained by aligning the
macroscopic average electrostatic potentials of the constituent slabs with that of the full interface. This method
accounts for relaxation, dipole formation, and interfacial reconstruction. In cases where electronic coupling is
strong, or the valence states are spatially delocalised across layers (as in type-II or type-III band alignment),
transport characteristics such as charge separation, carrier confinement, or tunnelling probability can change
significantly.

Additionally, charge transport properties such as electrical conductivity, Seebeck coefficient, or mobility may be
approximated from the band dispersion near the Fermi level using Boltzmann transport theory, or more rigorously via
non-equilibrium Green\rqss function (NEGF) methods, especially for low-dimensional or nanoscale interfaces.

Overall, both adhesion and electronic transport predictions depend on the level of theory and the nature of the
interface. While DFT remains the most widely used method for accuracy, its computational cost limits system size. For
larger systems or rapid screening, surrogate models such as machine-learned potentials or analytical approximations may
be employed, with the understanding that they must be carefully validated against reference calculations.


\section{Emergence of Machine Learning in Materials Science}
\label{section:machine_learning}

Over the past decade, machine learning (ML) has emerged as a transformative tool in materials modelling, offering a
route to accelerate predictions and expand accessible system sizes beyond those tractable by first-principles methods.
Traditional computational techniques, such as Density Functional Theory (DFT), Molecular Dynamics (MD), and Kinetic
Monte Carlo (KMC), have established themselves as accurate and rigorous, but they often scale poorly with system size
or configurational complexity. This limitation becomes particularly acute in the modelling of interfaces, where subtle
geometric, chemical, and electronic variations can dramatically affect material behaviour. ML offers a complementary
paradigm, enabling the construction of surrogate models trained on high-fidelity data, which can interpolate or even
extrapolate to predict key material properties across vast configuration spaces.

This section introduces the growing role of ML in materials prediction, highlighting its application in interface
modelling and the broader landscape of data-driven approaches. Subsequent subsections will examine the types of models
in use, including classical regressors and deep neural architectures, alongside the nature of training data, typical
input descriptors, and validation strategies. Special attention will be given to machine-learned interatomic potentials
(MLPs), particularly those with demonstrated success in interfacial contexts such as MACE and ChgNet. Finally, we
reflect on the benefits and ongoing limitations of ML methods in materials science, including issues of
transferability, uncertainty quantification, and integration into automated modelling pipelines.

\subsection{How has machine learning been introduced into materials prediction?}

Machine learning (ML) has been introduced into materials prediction as a complementary tool to traditional
first-principles methods, offering improved scalability, predictive flexibility, and the potential for workflow
automation. Its adoption has been driven by limitations in conventional approaches such as Density Functional Theory
(DFT), particularly in capturing the structural and chemical complexity of real-world systems such as interfaces.

\subsubsection{Computational Motivation}

First-principles methods scale poorly with system size, typically as \( \mathcal{O}(N^3) \), where \( N \) is the
number of electrons or basis functions. This makes simulations of large or disordered systems, especially those
involving interfaces with long-range strain relaxation, misfit dislocations, or extended defects, computationally
prohibitive. Machine-learned interatomic potentials (MLPs) such as MACE or ChgNet offer near-DFT accuracy at a fraction
of the cost, enabling simulations with thousands of atoms.

\subsubsection{Exploration of Complex Configuration Spaces}

ML enables efficient exploration of vast configurational landscapes, where brute-force DFT sampling would be
infeasible. This is particularly important in interface modelling, where multiple stacking orders, terminations, and
slip configurations can exist. ML-driven screening allows researchers to rapidly identify promising candidates for
further evaluation.

\subsubsection{Initial Integration Pathways}

Early ML applications focused on surrogate models trained to replicate DFT-derived energies, forces, or properties.
Descriptor-based regression models, using quantities such as coordination numbers, Bader charges, or bond lengths, were
employed to predict interfacial stability or band alignments. These approaches have evolved toward end-to-end learning
frameworks using physically-informed descriptors or graph-based representations.

\subsubsection{Workflow Integration}

ML is now embedded into iterative prediction--generation--evaluation cycles. For example, predicted interface
favourability informs structure generation (e.g. stacking vectors or surface terminations), which are then evaluated by
DFT or ML surrogates. The results feed back into model training, supporting a self-refining loop. Tools such as ARTEMIS
benefit from such integration.

\subsubsection{Modern Model Types}

More recent work employs graph neural networks (GNNs), such as MACE or NequIP, which are equivariant to Euclidean
transformations and thus well-suited for atomistic systems. These models learn directly from atomic positions and
species, avoiding the need for handcrafted features while preserving symmetry constraints essential for interface
modelling.

\subsubsection{Toward Autonomy}

Ultimately, ML enables a shift from static modelling pipelines to more autonomous workflows. When coupled with
structure-generation protocols and physical constraints, ML can be used not only to accelerate predictions but to guide
the discovery of novel interfacial phenomena in mixed-dimensional and disordered materials.

\subsection{What types of models and data are being used?}

Recent advances in machine learning (ML) have introduced a diverse range of models for predicting materials properties,
including those relevant to interface stability, ene.g. and structure. These models can be broadly categorised into
classical learning algorithms and deep learning approaches. Classical methods include random forests, gradient boosting,
support vector machines (SVMs), and Gaussian process regression (GPR), the latter being particularly valuable for small
datasets due to its uncertainty estimation capability. Deep learning architectures include feedforward neural networks
and, more notably, graph neural networks (GNNs), which are well-suited for materials modelling due to their ability to
represent variable-sized atomic graphs with complex bonding topologies.

The input to these models typically consists of numerical representations of atomic configurations known as
descriptors. Geometric descriptors include bond lengths, coordination numbers, angular distributions, $n$-body
distribution functions, and radial symmetry functions; many of which feature prominently in Behler-Parrinello-type
neural networks. Chemical descriptors such as electronegativity differences and Bader charges capture bonding character
and charge transfer. Electronic descriptors like partial density of states (PDOS) and local work functions are relevant
for predicting band alignment and interfacial transport properties. Interface-specific features, such as interfacial
roughness or vacuum level offsets, are particularly crucial in 2D|3D and mixed-dimensional heterostructures.

To support learning across such heterogeneous and high-dimensional features, symmetry-invariant and scale-independent
representations have become increasingly important. These ensure that models generalise across materials with different
atom counts or lattice sizes. Dimensionality reduction methods and learned embeddings (e.g. via autoencoders or message
passing) are frequently used to compress structural information into a lower-dimensional latent space optimised for
prediction.

The datasets used to train such models vary in size and fidelity. They often derive from high-throughput DFT
calculations available in public repositories such as the Materials Project or AFLOW. In cases targeting interfaces,
the training data may originate from internal simulations using structure generation pipelines like ARTEMIS or RAFFLE,
where candidate structures are pre-screened by DFT or empirical methods for energetic viability.

The combination of curated descriptors, structured data from known interfaces, and model architectures sensitive to
local and long-range atomic interactions underpins the growing success of ML in materials interface prediction.

\subsection{What are the strengths and limitations of ML in this field?}

Machine learning (ML) has emerged as a promising strategy for accelerating materials modelling, particularly in the
context of complex interfacial systems. Its strengths lie chiefly in efficiency, scalability, and adaptability. ML
models, once trained, can rapidly approximate formation energies, surface reactivities, and stability metrics for
thousands of configurations, many of which would be prohibitively expensive to assess using first-principles methods
alone. This computational efficiency enables exploration of broader configurational and chemical spaces, including
systems with disorder, stacking faults, or metastable structures that pose challenges for traditional simulations.

Another key advantage is the capacity of ML to capture subtle, non-linear correlations in high-dimensional datasets.
When used in conjunction with tools such as ARTEMIS or RAFFLE, ML can assist not only in predicting energetics, but
also in prioritising candidate geometries during structure generation. Moreover, deep learning approaches such as graph
neural networks (GNNs) and equivariant message-passing architectures have demonstrated good performance on systems with
varied local coordination environments, as frequently found at 2D|3D or lateral heterointerfaces.

However, ML approaches also exhibit critical limitations. A foremost concern is their dependence on the training
dataset: models typically interpolate well within the domain of known examples but generalise poorly to unseen
chemistries or extreme geometries. This limits their reliability when extrapolating to novel material classes or
interface types, unless training sets are curated with care to ensure coverage of relevant chemical and structural
motifs.

Another limitation is interpretability. While classical models (e.g. those based on density functional theory) yield
physically grounded outputs, such as charge densities or band structures, many ML models operate as statistical black
boxes. Though efforts are underway to integrate physically meaningful descriptors (e.g. n-body distribution functions,
local work function, or Bader charges), ensuring transparency and robustness remains a work in progress.

Finally, ML alone cannot fully replace physics-based methods. For systems exhibiting emergent interfacial phases,
electronic reconstructions, or coupled phonon-electron effects, high-fidelity methods like DFT or many-body
perturbation theory remain necessary for validation and refinement. As such, a hybrid strategy is often advocated;
using ML to pre-screen candidate structures and flag configurations for further high-accuracy assessment.

\paragraph{In summary} ML constitutes a valuable addition to the materials modelling toolkit, particularly for
screening and guiding interface prediction. Yet its performance remains contingent on data quality and physical
context, and it is best deployed in tandem with more rigorous electronic structure methods.

\subsection{Machine-Learned Interatomic Potentials (MLPs)}

Machine-learned interatomic potentials (MLPs) have emerged as a powerful alternative to classical force fields and
first-principles methods for atomistic simulations. By learning from high-fidelity data, typically generated from
Density Functional Theory (DFT), these models aim to reproduce the accuracy of ab initio methods while offering
orders-of-magnitude reductions in computational cost. This is particularly valuable for interface modelling, where
large supercells and long-range relaxation effects often render direct DFT approaches intractable.

\subsubsection{MACE}

The MACE (Many-body Attention and Correlation Equivariant) model represents a recent advancement in equivariant message
passing neural networks (MPNNs), tailored for high-accuracy and scalable force field construction. Unlike earlier MPNNs
which rely primarily on two-body invariant messages, MACE employs many-body interactions and equivariant tensor
contractions to capture angular and chemical complexity with improved expressivity and data efficiency.

One notable strength of MACE is its capacity to reach state-of-the-art accuracy on diverse benchmarks, including rMD17,
3BPA, and AcAc, while requiring only two message passing iterations, due to its use of high-body-order messages. For
instance, in extrapolation tasks involving out-of-distribution molecular geometries or high-temperature trajectories,
MACE (particularly with $L=2$) has demonstrated superior performance and robustness relative to competing models such as
NequIP and BOTNet.

Critically for materials modelling, MACE supports parallel evaluation across many GPUs by limiting its receptive field
through local message construction and efficient tensor operations. This makes it attractive for large-scale interface
studies involving thousands of atoms; especially where relaxation and strain distribution across the interface play a
pivotal role.

\subsubsection{ChgNet}

ChgNet is a neural network potential that incorporates both atomic positions and charge distributions into its
prediction framework, thereby explicitly learning charge transfer phenomena. This is particularly relevant for
interfaces where electrostatics, dipole formation, and local charge accumulation significantly affect electronic
properties such as band alignment.

While detailed benchmark results for ChgNet are less mature compared to MACE, its underlying design philosophy,
embedding local charge environments directly into the model, offers a complementary advantage. For instance, ChgNet
could be better suited to capturing interface-induced dipoles, charge spillover, or mixed-valence behaviour, which are
challenging to encode via purely structural descriptors.

\subsubsection{Use in Interface Modelling}

In the context of this project, MLPs like MACE and ChgNet offer distinct but synergistic advantages. MACE, with its
speed, high accuracy, and symmetry-respecting architecture, is well-suited for structure relaxation and force
prediction across multi-material junctions. ChgNet, on the other hand, may provide more faithful predictions of
electronic interface effects arising from local charge redistribution.

Nonetheless, these models are not without limitations. MLPs depend heavily on the representativeness of their training
data and can struggle to generalise to unseen chemistries or dissimilar interface environments. Therefore, hybrid
workflows that use MLPs for broad structural exploration, followed by targeted DFT recalculations for ambiguous
configurations, are increasingly being adopted.

The eventual goal is to integrate MLPs into an active learning pipeline alongside tools like RAFFLE or ARTEMIS,
facilitating the generation, screening, and optimisation of candidate interface structures at scale.


\section{Limitations of Current Approaches}
\label{section:limitations}

Despite notable advances in both physics-based and machine learning (ML) approaches to materials modelling, several
persistent limitations constrain the predictive fidelity, scalability, and generalisability of current methods;
particularly when applied to realistic interface systems. This section outlines three critical domains in which these
limitations manifest, motivating the hybrid methodologies explored later in this report:

\begin{itemize}
    \item \textbf{What are the current bottlenecks in interface prediction?} \\
    Interface prediction remains an inherently multiscale problem, combining combinatorial complexity at the atomic level
    with long-range elastic and electrostatic effects. First-principles approaches such as density functional theory
    (DFT) offer high accuracy but scale cubically with system size, making them infeasible for large or disordered
    systems typical of interfaces. Even with machine-learned interatomic potentials (MLPs), exploring configurational
    space, e.g. over terminations, interlayer slips, and misfit reconstructions, requires extensive sampling and often
    yields flat energy landscapes with many metastable minima.

    \item \textbf{Where do both ML and physics-based methods fall short?} \\
    DFT underestimates band gaps and struggles to capture emergent interfacial phenomena such as dipole formation, phase
    reconstruction, or charge transfer without recourse to hybrid functionals or many-body perturbation theory.
    Conversely, ML models, especially those trained on bulk or idealised configurations, exhibit limited transferability
    to chemically complex, strained, or disordered interfaces. Without embedded physical constraints or domain-specific
    descriptors, ML models often fail to resolve interfacial states, misfit dislocations, or partial intermixing.

    \item \textbf{Are there data or generalisation issues?} \\
    Interface-specific datasets are significantly less developed than those for bulk crystals, due in part to the vast
    configurational diversity of junctions and the lack of standardised repositories. This data sparsity hampers the
    ability of ML models to generalise across terminations, chemistries, or stacking motifs. Additionally, many early ML
    workflows lack symmetry-invariant and scale-independent representations, increasing the risk of overfitting to
    specific atom counts or lattice types.
\end{itemize}

These limitations underscore the need for hybrid workflows that integrate ML potentials (e.g. MACE, ChgNet) for
large-scale screening, DFT for high-fidelity validation, and structure-generation tools such as ARTEMIS or RAFFLE for
efficient sampling of the configurational landscape. Such frameworks aim to balance accuracy, efficiency, and
transferability in modelling the behaviour of realistic interfacial systems.

\subsection{What are the current bottlenecks in interface prediction?}

Despite considerable advances in computational materials science, reliable prediction of interface structures remains a
fundamental bottleneck. This difficulty stems from the intrinsic complexity of real interfaces, which are often
chemically heterogeneous, strain-mediated, and structurally disordered. The configurational degrees of freedom, ranging
from slip vectors, surface terminations, and intermixing patterns to stacking sequences and misfit accommodation,
create a vast structural landscape that resists exhaustive exploration via conventional methods.

\paragraph{Configurational Complexity.} At the atomic scale, even between lattice-matched materials, variations in
terminations, registry shifts, and interlayer relaxations produce an exponential number of possible configurations. The
search space rapidly becomes unmanageable, especially in systems requiring large supercells to capture long-range
strain or disorder. This renders brute-force density functional theory (DFT) calculations infeasible for comprehensive
screening.

\paragraph{Computational Cost.} DFT, while accurate and chemically grounded, exhibits cubic scaling with respect to the
number of electrons or basis functions, making simulations of interfaces with hundreds to thousands of atoms
prohibitively expensive. Moreover, structural relaxations in such systems often converge poorly due to shallow energy
gradients and multiple metastable minima. These numerical instabilities pose a challenge even for single configuration
evaluations.

\paragraph{Lack of Generalisable Descriptors.} Unlike bulk systems, for which symmetry and stoichiometry often impose
constraints, interfaces exhibit broken symmetries, compositional gradients, and non-uniform electrostatics. There is no
broadly applicable analogue to Anderson\rqss rule for predicting interfacial band alignment, necessitating ad hoc
corrections such as $\Delta E_\Gamma$ and $\Delta E_{\mathrm{IF}}$ that work only for narrow material classes.

\paragraph{Limitations of Current Structure Generation Tools.} Automated interface builders like ARTEMIS and RAFFLE
enable systematic generation of interface candidates but are typically restricted to heuristic approaches such as rigid
lattice matching or void-filling. These tools often fail to account for elastic mismatch, surface reconstructions, or
stacking instabilities. Furthermore, the lack of integration with energetic feedback loops results in large pools of
candidates requiring expensive post hoc evaluation.

\paragraph{Transferability of Machine-Learned Potentials.} Machine-learned interatomic potentials (MLPs), such as MACE
and ChgNet, offer pathways to accelerate structural screening at near-DFT accuracy. However, their performance is
contingent upon the representativeness of the training data. Interfacial environments frequently exhibit local
chemistries and coordination motifs absent from bulk-derived datasets. As such, model extrapolation can lead to large
errors, particularly in strained, defected, or reactive regimes.

\paragraph{Summary of Bottlenecks.}
\begin{itemize}
    \item The combinatorial complexity of plausible interface configurations.
    \item High computational cost and poor scaling of DFT-based relaxations.
    \item Absence of generalised physical descriptors for predicting interfacial properties.
    \item Limited adaptability of current structure generation frameworks.
    \item Transferability limitations of MLPs in interfacial chemical environments.
\end{itemize}

Overcoming these bottlenecks likely requires hybrid workflows that combine MLP-based structural exploration with
targeted DFT refinements, supported by uncertainty quantification and descriptor-led screening. Such frameworks are
increasingly positioned as a pathway to scalable, realistic, and physics-informed interface prediction.

\subsection{Where do both ML and physics-based methods fall short?}

Despite significant progress in both first-principles and machine learning (ML) approaches, key limitations persist in
the predictive modelling of material interfaces. These limitations arise not only from computational constraints but
also from intrinsic mismatches between real interfacial complexity and the assumptions underpinning current models.

Physics-based methods, most notably Density Functional Theory (DFT), provide high accuracy for local bonding
environments and band structure prediction. However, the computational scaling of standard Kohn-Sham DFT---typically
$\mathcal{O}(N^3)$ with respect to system size---renders large or disordered interface simulations intractable without
approximation. Even when feasible, DFT struggles to capture extended features such as long-range strain relaxation,
misfit dislocations, and spatially diffuse defect states that dominate realistic interface behaviour.

Machine learning potentials (MLPs), such as those based on graph neural networks or equivariant message passing
architectures, have emerged as promising tools to extend predictive power to larger systems at near-DFT accuracy.
Nonetheless, they too face structural limitations. MLPs require extensive, representative training datasets, typically
derived from DFT calculations, and their performance deteriorates when applied to configurations involving unseen
chemistries, surface reconstructions, or high strain regimes. Furthermore, many ML models lack rigorous uncertainty
quantification, making it difficult to detect failure modes in out-of-distribution predictions.

Both DFT and ML approaches also fall short in capturing emergent interfacial phenomena such as charge transfer,
interface-induced dipoles, or the formation of new interfacial phases. These effects are often strongly dependent on
the local atomic registry and chemical heterogeneity, and cannot be inferred from the properties of the isolated
constituents alone. Notably, simple models like Anderson\rqss rule have been shown to systematically fail in 2D and 2D|3D
heterostructures, necessitating corrections such as $\Delta E_\Gamma$ and $\Delta E_\mathrm{IF}$ that explicitly depend
on the atomic-scale interface structure and electrostatics.

\paragraph{In summary} while ML and physics-based models have complementary strengths, neither is sufficient in
isolation for the reliable prediction of interfacial behaviour. Their limitations motivate the development of hybrid
workflows, such as embedding physical constraints into ML architectures, or coupling MLP-based screening with selective
DFT refinement. These approaches aim to achieve a balance between scalability, accuracy, and generalisability---a
balance that remains at the forefront of interface modelling challenges.

\subsection{Are there data or generalisation issues?}

The predictive performance of machine learning (ML) models for interfacial systems is fundamentally constrained by data
availability, quality, and representational fidelity. While ML models have achieved notable success in bulk materials
prediction, their application to interface science introduces distinct challenges; chief among them data scarcity,
limited transferability, and inadequate descriptor design.

\subsubsection{Data scarcity and imbalance}

In contrast to bulk crystals, where structured repositories like the Materials Project or AFLOW provide extensive
coverage, interface-specific datasets remain sparse. This is due to the combinatorial explosion of viable terminations,
slip vectors, and orientations across material pairs, especially in mixed-dimensional systems. Existing databases tend
to overrepresent a narrow set of lattice-matched, technologically prominent interfaces, introducing bias and hindering
the model's capacity to generalise across broader chemical or structural classes.

todo

\subsubsection{Transferability and chemical diversity}

Machine-learned potentials (MLPs) such as MACE and ChgNet often fail to extrapolate to configurations that deviate from
the training distribution; particularly at strained, chemically heterogeneous, or electronically reconstructed
interfaces. The emergence of site-specific hybridisation, localised charge transfer, and dipole formation presents
physics not typically encountered in bulk datasets. Without explicit exposure to such environments during training,
even state-of-the-art equivariant models exhibit elevated uncertainty and reduced fidelity in these regimes.

\subsubsection{Descriptor and representation limitations}

Generalisation error is further amplified by suboptimal representation of interface-specific physics. Many descriptor
schemes, such as SOAP vectors or radial symmetry functions, are optimised for bulk periodicity and fail to capture
interfacial asymmetry, vacuum discontinuities, or broken symmetry. Representations that do not enforce scale-invariance
or rotational equivariance can also hinder model transferability to supercells or faceted configurations. Physically
meaningful descriptors, such as interfacial dipoles, local charge density, or cleavage ene.g. remain underused, though
they show promise for enhancing predictive robustness.

\subsubsection{Label quality and training signal noise}

Finally, inconsistencies in training labels present an often-overlooked source of predictive error. Energies obtained
from high-throughput DFT pipelines can vary depending on pseudopotential choice, relaxation tolerances, or convergence
thresholds. In interfacial contexts, these uncertainties are magnified by metastability and complex relaxation
pathways. Some datasets rely on static, unrelaxed geometries or surrogate models, further undermining the reliability
of ground truth energetics.

\paragraph{In summary} while ML offers substantial speed-ups over DFT, its utility in interface prediction is presently
constrained by data bottlenecks, representational mismatch, and transferability failures. Addressing these limitations
will require a multi-pronged strategy involving broader and more diverse training datasets, improved descriptors
attuned to interfacial physics, and integration of uncertainty quantification into ML workflows. Hybrid schemes, where
MLPs perform large-scale screening followed by selective DFT refinement, may offer a viable path forward.


\section{State-of-the-Art in Interface Prediction}
\label{section:state_of_the_art}

Recent years have seen rapid developments in the modelling of interface structures, driven both by methodological
advances and by the growing technological importance of heterostructured materials. Interface prediction now spans a
range of computational approaches, from first-principles calculations and high-throughput screening to data-driven and
machine learning (ML) methods. This section provides an overview of the current state-of-the-art, establishing the
context for the subsequent discussion of recent studies (Section 2.5.1), the positioning of this project\rqss methodology
(Section 2.5.2), and anticipated emerging directions (Section 2.5.3).

A key enabler of recent progress has been the development of automated interface generation tools such as ARTEMIS and
InterMatch, which systematically enumerate possible lattice matches, terminations, and stacking configurations between
two crystal structures. When integrated with rapid energy evaluation, either via density functional theory (DFT) or
machine-learned potentials (MLPs), these tools allow the configurational space of interface structures to be sampled
more broadly and with greater fidelity. In parallel, studies have increasingly sought to move beyond idealised,
coherent interfaces to include semi-coherent, faceted, and amorphous boundaries, reflecting the complexity of real
materials under fabrication conditions.

Efforts have also been made to improve the prediction of interfacial electronic properties, such as band alignment.
Where traditional rules like Anderson\rqss rule fail, corrective models based on physically grounded terms, such as
$\Delta E_{\Gamma}$ and $\Delta E_{\mathrm{IF}}$, have been proposed and validated across a range of 2D materials. Such corrections
are essential for capturing charge transfer and hybridisation effects at interfaces, and form an important bridge
between local structure and electronic response.

This chapter aims to synthesise these various advances, situating the present work within an evolving landscape of
interface modelling. Particular attention will be given to scalable workflows that combine structural generation,
energetic evaluation, and property prediction, setting the stage for a more autonomous and predictive framework for
materials interface discovery.

\subsection{What recent work has been done in this area?}

Recent years have witnessed a pronounced expansion in methodologies for predicting interface structures, driven by the
convergence of first-principles modelling, high-throughput screening, and data-driven inference.

This evolution reflects a shift from idealised, geometry-constrained models towards scalable frameworks capable of
capturing configurational complexity, long-range strain, and emergent phenomena at interfaces.

\subsubsection{First-principles structure prediction}

Hybrid methodologies now dominate the landscape of ab initio interface prediction. Notable among them is the
\textsc{ARTEMIS} tool developed by Taylor et al., which automates the generation of candidate interface structures by
identifying lattice-matched slabs, multiple surface terminations, and lateral shifts across a wide range of Miller
planes. This approach enables high-throughput sampling of plausible interfaces for subsequent relaxation using density
functional theory (DFT), bridging the gap between idealised registry-based models and defect-tolerant configurations.

Shortly after, Taylor et al.\ introduced the \textsc{RAFFLE}, with Pitfield et al.\ introducing methodology, combining
empirical void-filling algorithms with iterative statistical descriptors to model metastable and low-symmetry
reconstructions. Their work, particularly on graphene-encapsulated MgO, demonstrates that confinement and bonding at
the interface can stabilise novel phases, such as monolayer rocksalt MgO, not observed in bulk form.

\subsubsection{Search-based algorithms}

Global optimisation methods have been adapted to the interfacial context. Constrained variants of the Minima-Hopping
Method (MHM) allow for controlled exploration of grain boundary reconstructions by varying atomic densities and
introducing geometric constraints. These approaches yield more realistic reconstructions than fixed-stoichiometry DFT
or classical potentials, especially in systems like silicon where interface states are sensitive to atomic arrangement
and misfit accommodation.

\subsubsection{Machine learning integration}

Machine learning (ML) is increasingly embedded into the prediction workflow. Gerber et al.\ developed the
\textsc{InterMatch} framework, which leverages high-throughput DFT databases to train ML models that guide lattice
matching and stacking predictions in 2D systems. At the electronic level, Davies et al.\ proposed physically motivated
corrections to band alignment, namely $\Delta E_\Gamma$ and $\Delta E_{\mathrm{IF}}$, to extend Anderson\rqss rule and account for
hybridisation and interface dipoles in 2D heterostructures.

Moreover, the use of ML surrogates such as MACE and ChgNet has enabled large-scale relaxation of complex interfaces.
These interatomic potentials retain near-DFT accuracy while significantly lowering the computational cost, allowing
relaxation of thousands of atoms and exploration of non-ideal motifs including semi-coherent, faceted, or disordered
junctions.

\subsubsection{Ongoing challenges}

Despite these advances, several limitations persist. Data scarcity for interfacial systems hampers generalisation, and
most ML models are trained on bulk-like environments. Existing descriptors may fail to capture the broken symmetry,
local dipoles, and charge redistribution unique to interfaces. Furthermore, many workflows truncate the configurational
space prematurely or rely on geometric heuristics alone.

There is increasing recognition that hybrid workflows, combining ML surrogates with tools like \textsc{ARTEMIS} and
\textsc{RAFFLE}, and validating critical configurations using DFT, offer a scalable route forward. This multiresolution
strategy enables both efficient exploration and physically grounded evaluation of interfacial stability across diverse
systems.

\subsection{How does my proposed approach compare and build upon it?}

Contemporary approaches to interface structure prediction combine symmetry-based lattice matching, high-throughput
screening, and density functional theory (DFT) relaxation. Tools such as \textsc{ARTEMIS} enable the systematic
generation of interface candidates by identifying lattice commensurabilities and constructing multiple terminations and
stacking arrangements from parent surfaces. However, the energetic evaluation of these candidates remains a bottleneck:
full relaxation and total energy comparison using DFT is prohibitively expensive for systems exceeding a few hundred
atoms.

To mitigate this, recent advances have introduced surrogate strategies that bypass exhaustive DFT sampling. For
example, Davies \textit{et al.} proposed semi-empirical corrections (\(\Delta E_\Gamma\), \(\Delta E_{\mathrm{IF}}\)) to Anderson\rqss
rule for predicting band alignment in 2D heterostructures, based on descriptors derived from the constituent materials.
InterMatch, a high-throughput framework proposed by Gerber \textit{et al.}, guides stacking predictions using
pre-computed DFT databases and electrostatic heuristics. Yet such frameworks often lack spatial resolution or
generalisability across chemically diverse or defective systems.

This project builds directly upon these developments, advancing them in three ways:

\begin{enumerate}
    \item \textbf{Surrogate model integration.} Machine-learned interatomic potentials (MLPs), such as MACE, are
    employed to relax interfacial geometries with near-DFT accuracy but at significantly reduced cost. This enables
    structural optimisation of larger supercells that accommodate semi-coherent boundaries, strain gradients, or
    reconstructed motifs inaccessible to traditional DFT-only workflows.

    \item \textbf{Descriptor-led generalisation.} Inspired by the success of band-alignment corrections in TMDC
    heterostructures, this work seeks to identify whether interfacial stability can be inferred from properties
    computable on isolated surfaces; such as cleavage ene.g. surface dipole, or projected local density of states near
    the Fermi level. If validated, such descriptors would permit extrapolation to unseen terminations or chemistries
    without requiring full interface models.

    \item \textbf{Feedback into structure-generation pipelines.} In the longer term, this project aims to integrate
    ML-derived stability metrics into tools like \textsc{ARTEMIS} or \textsc{RAFFLE}, allowing early pruning of
    unfavourable configurations before computationally costly evaluation. This hybrid generation--relaxation loop
    reflects a broader trend towards data-assisted automation in interfacial materials modelling.
\end{enumerate}

The proposed framework thus combines multi-resolution structure prediction with predictive modelling rooted in
physically meaningful descriptors. Rather than replacing existing DFT-centric approaches, it seeks to augment them; by
embedding surrogate models within the structure-generation workflow and using them to navigate large configurational
spaces. This enables a shift from retrospective evaluation to prospective design of realistic, energetically favourable
interfaces.

\subsection{Emerging trends}

Recent advances in interface prediction have increasingly centred on integrating data-driven approaches with atomistic
modelling frameworks. Machine-learned potentials (MLPs) such as MACE and E(3)-equivariant networks like NequIP are now
enabling the efficient evaluation of interfacial configurations across large configurational spaces. This shift permits
relaxation of structures at near-DFT accuracy, allowing researchers to explore interface separations, registry shifts,
and metastable motifs that would be prohibitively expensive using standard ab initio methods.

Alongside these developments, tools like ARTEMIS and RAFFLE have emerged to systematically generate candidate
interfaces. These platforms sample multiple surface terminations and stackings while accounting for lattice mismatch,
enabling more representative modelling of real, faceted, or incoherent boundaries. The capacity to generate and filter
hundreds of interfaces also creates opportunities for coupling with high-throughput screening and surrogate models.

Emergent trends also include the move toward physically motivated corrections for electronic properties at interfaces.
For instance, the failure of Anderson\rqss rule in 2D heterostructures has prompted the introduction of corrective terms
such as $\Delta E_\Gamma$ and $\Delta E_\mathrm{IF}$, which better account for hybridisation and interface dipoles. These terms enable
more accurate band alignment predictions across families of 2D materials without requiring full-scale DFT recalculation
for each pair.

Collectively, these approaches suggest a field increasingly focused on scalable, automatable, and physically
interpretable modelling of interfaces. The combination of machine learning with established electronic structure theory
is opening a path toward predictive frameworks that are sensitive to real-world synthesis constraints and device
contexts.

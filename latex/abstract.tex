\begin{abstract}
Material interfaces underpin the performance of a wide range of functional devices, from microelectronic transistors to
catalytic systems. Their properties often emerge from interfacial effects not captured by bulk characterisation, making
the prediction of interface stability a critical and computationally demanding task. While Density Functional Theory
(DFT) provides accurate energetics, its high cost limits its applicability to large-scale screening. This study
investigates whether machine-learned interatomic potentials (MLPs), specifically the MACE framework, can replicate
DFT-derived rankings of interface favourability across a diverse set of semiconductor heterostructures.

Interfaces were generated using the ARTEMIS toolkit across three compositional stages: initial Si|Ge binaries, expanded Si|Ge
binaries, and a broader C-Si-Ge-Sn set. Each interface structure was independently relaxed using both DFT and MACE.
Energies were compared using rank correlation metrics (Spearman $\rho$, Kendall $\tau$), absolute error measures (RMSE, MAE),
and Top-$N$ overlap to assess rank preservation.

Results indicate that MACE\rqs predictive performance is system-dependent. Rank correlation was weak for silicon-only
interfaces but improved markedly for chemically ordered alloy systems, such as Si|Ge. Cross-method evaluations suggested
that structural relaxation, rather than energy evaluation, is the primary source of deviation between DFT and MLP
predictions.

Future work will focus on training delta-learning models to correct systematic MLP errors using DFT data and on
developing a predictive machine-learning interface (MLI) model based on slab features. The role of surface orientation
and structural features in determining interface stability will also be examined.
\end{abstract}
